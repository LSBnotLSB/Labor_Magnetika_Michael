\documentclass[a4paper,twoside,12pt,DIV=13,BCOR=5mm,numbers=noenddot,cleardoublepage=empty]{scrbook}
%======= Einbinden der benötigten Packete (= Zusatzfunktionen)
\usepackage[T1]{fontenc}                % für Fonts in westeuropäischer Codierung
\usepackage{lmodern}										% Latin Modern Paket verändert die verwendete Schriftart. Bessere Darstellung für pdf
\usepackage{textcomp}										% Provides extra symbols, e.g. arrows like \textrightarrow, various currencies (\texteuro,..
\usepackage[latin1]{inputenc}    				% german special characters
\usepackage[english,ngerman]{babel}     % hyphenation   usepackage[english,ngerman]{babel} f. deutsch
\usepackage{pdfpages}										% Einbinden von pdf-Files
\usepackage{pifont,textcomp,mathcomp}   % dingbad psfonts and text-compilant fonts (euro, TM, ...)
\usepackage{amsmath,amsopn,amsthm}      % AMS mathematics
\usepackage{amssymb}                    % zusätzliche Symbole 
\usepackage{xspace}											% avoids eaten spaces

%======= Eine Umgebung für Bilder und Tabellen			  	
%\usepackage[textfont={Small},labelfont={bf},margin=1cm,format=plain,font=singlespacing]{caption}   % hanging caption text [hang]
\usepackage[textfont={small},labelfont={bf},margin=1cm,format=plain,font=singlespacing]{caption}   % hanging caption text [hang]
\captionsetup*[figure]{name=Abb.}			 % Abbildungsunterschrift beginnt mit Abb.
\captionsetup*[table]{name=Tab.}


%======= Farben für Überschriften
\usepackage{color}
\definecolor{TUBlau}{rgb}{0,0.4,0.6}   % TU-blau RGB 0 102 153
\addtokomafont{sectioning}{\sffamily\bfseries\selectfont\color{TUBlau}}
\setkomafont{chapter}{\normalfont\huge\sffamily\bfseries\color{TUBlau}}
\addtokomafont{section}{\Large}
\addtokomafont{subsection}{\normalfont\Large\sffamily\bfseries\color{TUBlau}}
\addtokomafont{subsubsection}{\normalfont\large\sffamily\bfseries\color{TUBlau}}
\addtokomafont{paragraph}{\normalfont\large\sffamily\bfseries\color{TUBlau}}


%======= Kopf-/Fußzeilen
\pagestyle{plain} % nur Fußzeile



%======= Eine kompakte Umgebung für die Bilder
\newcommand{\bild}[4]{{
\begin{figure}[#2]
\begin{center}
\includegraphics[scale=#3]{pictures/#1}
\caption{#4}\label{fig:#1}
\end{center}
\end{figure}
}}


%======= Definitionen eigener Befehle
\newcommand{\degC}{\ensuremath{^{\circ}}C}        % Grad Celsius
\newcommand{\Gu}{\glqq{}}                         % Gänsefüßchen unten
\newcommand{\Go}{\grqq{}\xspace}    												% Gänsefüßchen oben











 
\usepackage{ucs} 	
\usepackage{float}		% File mit den ben�tigten Packeten, den Formatanweisungen und den Befehlsdefinitionen
\begin{document}
\newcommand\todo[1]{\textcolor{red}{#1}}
%=============================================================================================
% Titelblatt und Inhaltsverzeichnis
%=============================================================================================
\renewcommand{\baselinestretch}{1.25}
\newcommand{\StudentA}{Marton Harsch}
\newcommand{\MatrNrA}{12123680}
\newcommand{\StudentB}{Michael Malburg}
\newcommand{\MatrNrB}{61806515}
\newcommand{\StudentC}{Jonathan Gamperl}
\newcommand{\MatrNrC}{12302766}

\newcommand{\LUDatum}{22.4.2024}
\newcommand{\LUGruppe}{}
\newcommand{\LUBetreuer}{}

\large
\includepdf[fitpaper=true,
						picturecommand*={\unitlength1cm 
						\put(7.3,7.7){\StudentA} \put(14.1,7.7){\MatrNrA}
            \put(7.3,7.0){\StudentB} \put(14.1,7.0){\MatrNrB}
            \put(7.3,6.3){\StudentC} \put(14.1,6.3){\MatrNrC}
						\put(7.3,5.1){\LUDatum} 
            \put(7.3,4.4){\LUGruppe} 
            \put(7.3,3.7){\LUBetreuer} 
}]
{pictures/DeckblattLUDie}     %file name of title page


%===============================================================================
% Text
%===============================================================================

\cleardoublepage
\setcounter{tocdepth}{3}

\setcounter{page}{0}
\renewcommand{\thepage}{\roman{page}}
\tableofcontents \cleardoublepage

\setcounter{page}{1}
\renewcommand{\thepage}{\arabic{page}}
\setcounter{chapter}{0}


%===============================================================================

\newpage
\chapter{Zusammenfassung}
Die Zusammenfassung
\newpage
\chapter{Ringkerntrafo}
\section{Aufnahme einer Hystereseschleife}

\section{Aufnahme der Permeabilitätskurve}


\section{Hystereseschleife und Entmagnetisierung}


\section{Die Neukurve}

\section{Fazit}

\chapter{Übungen am Elektromagneten}
\section{Metallscheiben im Magnetfeld}

\section{Eisenblech im Magnetfeld}

\section{Diamagnetische Stoffe}

\section{Messung des Magnetfeldes}

\section{Lorentzkraft}

´\section{Fazit}



\subsection{Messung} 



\end{document}


